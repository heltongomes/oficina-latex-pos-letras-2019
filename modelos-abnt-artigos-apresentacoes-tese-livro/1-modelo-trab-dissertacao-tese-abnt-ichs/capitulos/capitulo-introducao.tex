% ----------------------------------------------------------
% Introdução (exemplo de capítulo sem numeração, mas presente no Sumário)
% ----------------------------------------------------------
\chapter[Introdução]{Introdução}
%\addcontentsline{toc}{chapter}{Introdução}
% ----------------------------------------------------------

\section{Justificativas e Relev{\^a}ncia}
%
Este documento e seu código-fonte são exemplos de referência de uso da classe
\textsf{abntex2} e do pacote \textsf{biblatex-abnt}. O documento exemplifica uma realização possível entre as opções existentes na norma ABNT NBR 10520:2018 \emph{Citações em documentos -- Apresentação} e da norma ABNT NBR 6023:2018 \emph{Referências -- Elaboração}, cientes de que existe uma distância entre as ``normas'' e a interpretação das normas. Assim, antes de tudo, converse com seu orientador ou representantes do programa de pós-graduação de sua universidade, mostre uma cópia do documento PDF gerado por este arquivo e certifique-se de que não terá problas futuros com relação à aceitação ou não do modelo.

A expressão ``Modelo Canônico'' é utilizada para indicar que \abnTeX\ não é modelo específico de nenhuma universidade ou instituição, mas que implementa tão somente os requisitos das normas da ABNT. 

\section{Metodologia}
Comecemos então com um exemplo de citação, como esta aqui, feita em notas explicativas,\footcite[Esta é uma nota explicativa. Cf. e.g.,][\S 12]{boyle1772} conforme a norma NBR 10520:2018, existem dois tipos de citações em notas de rodapé:
\begin{citacao}
	\begin{itemize}
		\item[$3.6$] \textbf{notas de rodapé:} Indicações, observações ou aditamentos ao texto feitos pelo autor, tradutor ou editor, podendo	também aparecer na margem esquerda ou direita da mancha gráfica. 
		\item[$3.7$] \textbf{notas explicativas:} Notas usadas para comentários, esclarecimentos ou explanações, que não possam ser incluídos no texto.
	\end{itemize} 
\end{citacao}


Uma outra citação nada a ver.\footcite[Esta é uma outra nota explicativa. Ver também ][p.~12]{herao}
%Uma outra citação nada a ver.\cite[Esta é uma outra nota explicativa. Ver também ][p.~12]{herao}. Ver também \cite{boyle1772}

%\caixa{O autor não é obrigado a seguir as notas explicativas com o comando ``footcite'', que gera a essa nota explicativa.\footcite{boyle1772} Pode utilizar o estilo autor-data igualmente, mas deve seguir o mesmo padrão ao longo do texto. Por exemplo, se usar o comando ``cite'', será gerada a entrada \cite{descartes-oeuvres-volx} ou, se quiser apenas o autor, com o comando ``citeauthor'', será gerada a entrada\citeauthor[][12]{descartes-oeuvres-volx}. Ainda, se quiser imprimir apenas a data de referida obra, basta utilizar o comando ``citedate'', que gera a entrada \citedate[55]{descartes-oeuvres-volx}.}

\lipsum[1]

\section{Objetivos}

\lipsum[7]

\lipsum[8]

\section{Organiza{\c c}{\~a}o e estrutura}

\lipsum*[9-11]

\begin{itemize}
\item item;
\item item;
\item item;
\item item;
\end{itemize}

\section{Cronograma}

Esta seção deve constar somente no projeto de monografia. Não deve aparecer na versão final do texto.

% Please add the following required packages to your document preamble:
% \usepackage[table,xcdraw]{xcolor}
% If you use beamer only pass "xcolor=table" option, i.e. \documentclass[xcolor=table]{beamer}

% Please add the following required packages to your document preamble:
% \usepackage[table,xcdraw]{xcolor}
% If you use beamer only pass "xcolor=table" option, i.e. \documentclass[xcolor=table]{beamer}

Um exemplo de cronograma das atividades é proposto na tabela \ref{tab:cronograma}.\footnote{Voc{\^e} pode elaborar também tabelas online, gerando o código em \LaTeX. Após isso, basta copiar e colar o código aqui. Um exemplo de site é o ``Table Generator''\url{http://www.tablesgenerator.com/}.}
  
\begin{table}[h]
\ABNTEXfontereduzida
\caption[Cronograma das atividades]{Cronograma das atividades de elaboração da monografia.}
\label{tab:cronograma}
\begin{minipage}{0.3\textwidth}
    \centering
\begin{tabular}{|l|l|l|l|l|l|l|l|l|l|l|l|l|l|l|l|l|}
\hline
                             & \multicolumn{16}{c|}{Meses}                                                   \\ \hline
Atividades (Etapas)          & 01 & 02 & 03 & 04 & 05 & 06 & 07 & 08 & 09 & 10 & 11 & 12 & 13 & 14 & 15 & 16 \\ \hline
1. Estudo da teoria          & X  & X  & X  & X  & X  &    &    &    &    &    &    &    &    &    &    &    \\ \hline
2. Atualização bibliográfica & X  & X  & X  & X  & X  & X  & X  &    &    &    &    &    &    &    &    &    \\ \hline
3. Seleção de Material       & X  & X  & X  & X  & X  & X  & X  & X  & X  &    &    &    &    &    &    &    \\ \hline
4. Elaboração da monografia  &    &    &    &    & X  & X  & X  & X  & X  & X  & X  & X  & X  & X  & X  &    \\ \hline
5. Elaboração de Artigo      &    &    &    &    &    &    &    &    & X  & X  & X  & X  & X  & X  & X  &    \\ \hline
6. Defesa da monografia      &    &    &    &    &    &    &    &    &    &    &    &    &    &    &    & X  \\ \hline
\end{tabular}
  \end{minipage}
\end{table}